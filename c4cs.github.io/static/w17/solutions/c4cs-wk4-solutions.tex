\documentclass{article}
\usepackage[T1]{fontenc}

\usepackage{amssymb}
\usepackage{courier}
\usepackage{fancyhdr}
\usepackage{fancyvrb}
\usepackage[top=.75in, bottom=.75in, left=.75in,right=.75in]{geometry}
\usepackage{graphicx}
\usepackage{lastpage}
\usepackage{listings}
\lstset{basicstyle=\small\ttfamily}
\usepackage{mdframed}
\usepackage{parskip}
\usepackage{soul}
\usepackage{tabularx}
\usepackage{textcomp}
\usepackage{upquote}
\usepackage{xcolor}

% http://www.monperrus.net/martin/copy-pastable-ascii-characters-with-pdftex-pdflatex
\lstset{
upquote=true,
columns=fullflexible,
keepspaces=true,
literate={*}{{\char42}}1
         {-}{{\char45}}1
         {^}{{\char94}}1
}
\lstset{
  moredelim=**[is][\color{blue}\bf\small\ttfamily]{@}{@},
}

% http://tex.stackexchange.com/questions/40863/parskip-inserts-extra-space-after-floats-and-listings
\lstset{aboveskip=6pt plus 2pt minus 2pt, belowskip=-4pt plus 2pt minus 2pt}

\usepackage[colorlinks,urlcolor={blue}]{hyperref}
\usepackage[capitalise,nameinlink,noabbrev]{cleveref}
\crefname{section}{Question}{Questions}
\Crefname{section}{Question}{Questions}

\begin{document}

\fancyhead[C]{\hl{Select the right page in Gradescope or we will not grade the question!}}
\fancyhead[L]{}
\fancyhead[R]{}

\fancyfoot[L]{\color{gray} C4CS -- W'17}
\fancyfoot[R]{\color{gray} Revision 1.0}
\fancyfoot[C]{\color{gray} \thepage~/~\pageref*{LastPage}}
\pagestyle{fancyplain}


\title{\textbf{Homework 4\\Editors}}
\author{\textbf{\color{violet}{Solutions}}}
\date{}
\maketitle


\section*{Submission Instructions}
Submit this assignment on \href{https://gradescope.com/courses/3499}{Gradescope}.
You may find the free online tool \href{https://www.pdfescape.com}{PDFescape}
helpful to edit and fill out this PDF.
You may also print, handwrite, and scan this assignment.


\section*{\emph{Optional} Readings}

{\small

\subsection*{Interrupting Developers and Makers vs Managers}
In lecture we spent a little time talking about the mental model of
programmers. As computer science grows, there is a growing body of research
and wisdom in how to maximize the efficacy of individual programmers.
One of the key revelations is that programming is a creative process. It
requires time to ``page in'' what you are working on. As a result, even small
interruptions (read: text messages) can be far more costly than you may
realize.

\medskip
\noindent
\url{http://thetomorrowlab.com/2015/01/why-developers-hate-being-interrupted/}\\
\emph{Why developers hate being interrupted}, by Derek Johnson at The Tomorrow Lab.

\noindent
\url{http://www.paulgraham.com/makersschedule.html}\\
\emph{Maker's Schedule, Manager's Schedule}, by Paul Graham, co-founder of Y~Combinator.

\medskip
\noindent
Once you are in a job environment and meetings become a regular thing, one of
the best tricks I learned is to schedule a few 4-hour meetings with yourself
throughout the week. It both prevents others from interrupting you (your
calendar shows you as busy) and lets you mentally prepare for a good, reliable
work session.
(A similar alternative, some find
\href{http://lifehacker.com/productivity-101-a-primer-to-the-pomodoro-technique-1598992730}{pomodoro}
helpful)

%As you work on homework this week, especially for your programming-heavy EECS
%classes, I encourage you to try turning off your phone (yes, seriously, off),
%close your browser and start from a blank session (hiding the browser window
%with your Facebook tab doesn't help if messenger keeps dinging), and see if
%your software development experience is any different.\footnote{
%  It can be hard to be not keep opening a Facebook tab every time you Google
%  something. Sometimes when I am having a particularly hard time forcing
%  myself to work, I'll ``black hole'' things I might be tempted to distract
%  myself with: \\
%  \texttt{printf "\# Disable some sites\char`\\n127.0.0.1\char`\\tfacebook.com\char`\\n127.0.0.1\char`\\treddit.com\char`\\n" | sudo tee -a /etc/hosts}\\
%  You can undo this by editing \texttt{/etc/hosts} later.
%}

\subsection*{Software Engineering in the Real World}
\url{http://blogs.msdn.com/b/peterhal/archive/2006/01/04/509302.aspx}\\
\emph{What Do Programmers Really Do Anyway?}, by Peter Hallam, then-Microsoft~Developer
\begin{quote}
  \emph{%
    Why is 5 times more time spent modifying code than writing new code? The
    answer is that new code becomes old code almost instantly.  Write some new
    code. Go for coffee. All of sudden you've got old code.
  }
\end{quote}
This post gives a nice perspective on what enterprise software engineering is
really like. School projects are misleading. Rarely in life will you be faced
with a spec and a blank text file. Far more often you are building on or
improving something that came before you and will exist long after you.

\medskip
\noindent
\url{http://www.joelonsoftware.com/articles/fog0000000069.html}\\
\emph{Things You Should Never Do, Part I}, by Joel Spolsky
\begin{quote}
  \emph{It's harder to read code than to write it.}
\end{quote}
Building on the previous, rarely in life are you handed a spec and a blank text
file, often in life you are handed a spec and a pile of (seemingly) spaghetti
code that does at least some of it. This article discusses the hidden value of
the built-up institutional knowledge and the high value of working code.

\medskip
\noindent
\emph{When was the last time you saw a hunt-and-peck pianist?} -Jeff Atwood,
co-founder of Stack Overflow and Discourse.\\
Some extras on the
\href{http://blog.codinghorror.com/we-are-typists-first-programmers-second/}
{value of being a good typist}
and
\href{http://blog.codinghorror.com/going-commando-put-down-the-mouse/}
{ditching the mouse for the keyboard}.

}

% I had some intention of some articles on text editors here, but I think these
% are actually more useful, if a bit preachy


\newpage
\section{Trying out some of the fancy features}

People often complain about how difficult the simple things are using
text-based text editors. Indeed, Notepad (or TextEdit) make changing some
text and saving a file very easy.
In this question we explore some of the things that Notepad (and nano, and
gedit, and Microsoft Word\dots) cannot do.

I encourage you to play around with things as you work through this question.
The intent is to expose you to features you may not have been aware of along
with some context for why they are useful.

There are many things to try here.
\textbf{For credit on this question, you only need to choose any 5.}
That said, I encourage you to try all of these.

For this question, you may choose \texttt{vim} or \texttt{Emacs}, whichever
you are more comfortable with.

\begin{quote}
  \color{blue}I'll present solutions in blue for vim, \color{red}and in red for Emacs.
\end{quote}

\medskip
\noindent
To start, grab copies of two files full of code:
\begin{quote}
  \texttt{\# Examples from \url{http://web.mst.edu/~price/cs53/code_example.html}}\\
  \texttt{wget \url{http://web.mst.edu/~price/cs53/fs11/workDecider.cpp}}\\
  \texttt{wget \url{http://web.mst.edu/~price/cs53/KatsBadcode.cpp}}
\end{quote}
\smallskip
\hrule
\bigskip

\begin{enumerate}
  \item Sometimes you will come across some ugly code. Your editor can help
    make it better. Open \texttt{KatsBadcode.cpp}. Among other issues, the
    really bad tabbing makes this hard to read.\\
    \textbf{Describe how to automatically fix the whitespace for the whole file.}
    \begin{quote}\tt
      {\color{blue} gg=G (gg-go to top of file, = re-indent from here to, G-the end of the file)}

      {\color{red} C-x h <tab>} or {\color{red}C-x h C-M-\textbackslash}
      and as a courtesy to your collaborators, \newline give {\color{red} M-x delete-trailing-whitespace} a go
    \end{quote}
  \item Editing one file is nice, but often it's really useful to compare
    files side-by-side (source and headers, spec and implementation).\\
    \textbf{Describe the command sequence to create another window side-by-side with your current window, switch to it, and then open a file in that window.}
    \begin{quote}\tt
      {\color{blue} :split [filename] (or :vsplit), Ctrl-W + (W or arrow)}

      {\color{red} C-x 4 f <filename>} or more manually {\color{red} C-x 3 (or C-x 2), C-x o, C-x C-f <filename>}
    \end{quote}

    Try playing around with these two views, copy/paste code between them,
    bind them so they scroll together, resize them, add more splits.\\
  \item Editors also have pretty nice integration with compilation tools. With
    \texttt{KatsBadcode.cpp} open, try typing \texttt{:make KatsBadcode} in
    \texttt{vim} or \texttt{M-x compile<enter>make KatsBadcode} in
    \texttt{Emacs}.\footnote{
      Descriptions of \texttt{Emacs} commands are usually written as
      \texttt{C-c} or \texttt{M-x}, which mean ``Ctrl+C'' or ``Meta+x''
      respectively. Meta is often mapped to the escape and/or alt key,
      \texttt{M-x} means press Escape and then x or hold alt and press x.
    }
    \begin{quote}
      First cool thing that happened: Yes, you can run \texttt{make}
      \emph{without} any Makefile anywhere. We'll cover how that happened
      during build-system week.
    \end{quote}
    Building \texttt{KatsBadcode.cpp} will fail with several errors.\\
    \textbf{Describe how to navigate between compilation errors automatically.}
    \begin{quote}\tt
      {\color{blue} :cn, :cp}

      {\color{red} C-x `, M-g p} or {\color{red} M-x next-error, M-x previous-error}
    \end{quote}
  \item While C-style languages support \texttt{/* block comments */}, others
    such as Python have no block comments and require you to put a \texttt{\#}
    at the beginning of every line.\\
    \textbf{Describe how to efficiently comment out a large block of Python code.}
    \begin{quote}\tt
      \color{blue} Ctrl-v, highlight range, Shift-i, \#, Escape

      \color{red} C-space, highlight range, M-;
    \end{quote}
  \item (This one is actually about the terminal emulator): The normal
    keyboard shortcuts to copy/paste are Ctrl-C and Ctrl-V. These do not work
    in the terminal, however.\\
    \textbf{Explain what Ctrl-C does in a terminal.}
    \begin{quote}\small
      \color{violet} In a terminal, Ctrl-C sends the signal \texttt{SIGKILL} to
      the foreground process (whoa, wordy!).

      Breaking that down, normally when you write programs, your program asks
      the operating system for input events. If a user types ``hello'', your
      program doesn't see it until it calls \texttt{cin} (on the flip side, if
      your program calls \texttt{cin} before the operating system has anything
      to send it (before the user has typed anything), then your program is
      put to sleep by the operating system until there is data available).

      Sometimes, however, the operating system needs to send an
      \emph{asynchronous} message to your program, that is it needs to send a
      message that your program didn't ask for. In unix, \emph{signals} are
      the mechanism that deliver these messages. One such asynchronous message
      is the kill signal (\texttt{SIGKILL}), which is usually understood to
      mean that the user wants your program to exit. If your program does not
      implement a \emph{signal handler} (special function) and register it to
      catch \texttt{SIGKILL}, then the operating system will kill your program
      automatically (this is why Ctrl-C will kill the programs you write).

      Many programs will install handlers, however, to \emph{catch} the signal
      and do something different with it. For example, try hitting Ctrl-C when
      running vim, what happens? Ctrl-Z also sends a signal, \texttt{SIGSTOP}.
      This signal is special in that it \emph{cannot be caught}. When a
      program receives a \texttt{SIGSTOP} signal, it is immediately suspended.
      This can be a useful way to bail out of a program (such as \texttt{ed})
      that Ctrl-C isn't killing and you can't figure out how to stop.
      Note the program is \textbf{not} dead (yet), but you can kill a stopped
      program to remove it completely.
    \end{quote}
    \textbf{Describe how to copy/paste using the keyboard in a terminal.}
    \begin{quote}\tt
      \color{violet} Ctrl-Shift-C and Ctrl-Shift-V
    \end{quote}
  \item The default cut/copy/paste(/put/yank/kill) operations do not put text in
    the system clipboard (that is you cannot copy/paste into/from other
    applications).
    \textbf{Describe how to copy/paste a region of text to/from the system
    clipboard using your text editor.}
    \begin{quote}\tt
      \color{blue} Use "+ to select the system clipboard register (+), that is
      "+yy to copy the current line, or "+p to paste

      \color{red} A freebie because it basically works already, Copy M-w,
      Paste C-y -- however, I couldn't find an obvious way to make it work in
      -nw mode, which could cause issues if something spawns emacs without a
      window.
    \end{quote}
  \item Many times you have a repetitive task that you need to do say 10 or 20
    times. It's too short to justify writing a script or anything fancy, but
    it's annoying to type the same thing over and over again. As example,
    reformatting the staff list to a nice JSON object:
    \begin{verbatim}                                staff = [
Pannuto,Pat                        {"first": "Pat", "last": "Pannuto"},
Darden,Marcus                      {"first": "Marcus", "last": "Darden"},
Terwilliger,Matt                   {"first": "Matt", "last": "Terwilliger"},
Chojnacki,Alex                     {"first": "Alex", "last": "Chojnacki"},
                                ]\end{verbatim}
    Editors support the record and reply of \emph{macros}. With macros, you
    can start recording, puzzle out how to turn one line into the other, and
    then simply replay it for the rest. As a general rule, do not bother
    trying to be optimal or efficient, just find anything that works.
    Try starting with the column on the left and devising a macro to turn it
    into the column on the right.\\
    \textbf{Describe how to record and replay a macro.}
    \emph{\small (you do not need to include the contents of your macro)}
    \begin{quote}\tt
      \color{blue}qq, <commands>, q (to stop recording), @q to replay (note
      the second q is a register, so you could also do qa and @a)

      {\footnotesize(my macro): \verb!yypws^M^[i{"first": ^[A,^[kddpkJli"last": ^[A}^[0f:lli"^[wwi"^[f:lli"^[$i"^[kJ0j!}

      \color{red} C-x ( , <commands>, C-x ) , C-x e to replay

      {\footnotesize(my macro): <tab> {"first" <space> C-right C-f C-space C-right
          \newline C-w C-left " C-y ", <space> "last": <space> " C-right "}}
    \end{quote}
  \item Similar to how \texttt{$\sim$/.bashrc} configures your shell, a
    \texttt{$\sim$/.vimrc} or \texttt{$\sim$/.emacs} file\footnote{
      These files do not exist by default, you have to create them.
    } can configure how your editor behaves. Here are some excerpts from the
    staff's configuration files:
    {\color{blue}
    \begin{lstlisting}
    .vimrc:
    set number    " double-quote means comment in vimscript; this turns on line numbers
    map ; :       " this line and the next makes ; act like : so that you don't have to
    noremap ;; ;  " hit shift all the time, typing ";;" will act as ";" used to
    \end{lstlisting}
    }

    {\color{red}
    \begin{lstlisting}
    .emacs:
    ; line numbers - in Emacs, ; means a comment
    (global-linum-mode t)
    (setq linum-format "%d ")
    ; Changes all yes/no questions to y/n type
    (fset 'yes-or-no-p 'y-or-n-p)
    \end{lstlisting}
    }
    \textbf{Add something useful not listed above to your editor's configuration file.\\
      Describe  what you added and why.}
      \begin{quote}\tt\small
        \color{blue}
        " Show whitespace characters\\
        " helps notice bad tabbing and eliminates trailing whitespace characters\\
        set list\\
        set listchars=tab:.\ ,trail:.\\

        \color{red}
        ;set return go to newline and indent\\
        (global-set-key (kbd "RET") 'newline-and-indent)\\

        ; m-x compile scrolls automatically\\
        (setq compilation-scroll-output 'first-error)\\

        ; show trailing whitespace\\
        (setq-default show-trailing-whitespace t)\\
        ; delete trailing whitespace on save\\
        (add-hook 'before-save-hook 'delete-trailing-whitespace)\\

        ; enable better package manager\\
        (require 'package)\\
        (add-to-list 'package-archives\\
                     '("marmalade" . "http://marmalade-repo.org/packages/"))\\
        (add-to-list 'package-archives\\
                     '("melpa" . "http://melpa.org/packages/") t)\\
        (package-initialize)\\
        ; M-x install-package or M-x package-list-packages\\

      \end{quote}
    \item Some final little things
    \small
    \begin{enumerate}
      \item Editors understand balanced ()'s and \{\}'s.
        \textbf{Describe how to jump from one token to its match, such as from
          the opening \{ on line~29 of \texttt{KatsBadcode} to its partner \}
          on line~58.
        }
        \begin{quote}\tt
          \color{blue}Simply type \%

          \color{red} C-M-right, C-M-left
        \end{quote}
      \item Now imagine if you are editing code and you determine you can
        remove an if block, for example removing the \texttt{if} on line~28 of
        \texttt{KatsBadcode} and running its body unconditionally. You need to
        fix the indentation level of all 20~lines of the body.\\
        \textbf{Describe how to change the indentation level of a block of code.}
        \begin{quote}\tt
          \color{blue} Select the region (Shift-V), hit $<$ (or $3<$ for three in, etc)

          \color{red} C-u - 20 C-x <tab>
        \end{quote}
      \item Sometimes it's useful to run quick little commands. For example,
        your code needs to read from a file, but you can't remember if it's
        called \texttt{sample\_data.txt} or \texttt{sampleData.txt}.\\
        \textbf{Describe how to run `ls' directly from within your editor.}
        \begin{quote}\tt
          {\color{blue} :!ls}

          {\color{red} M-! ls} or {\color{red} M-x shell-command}
        \end{quote}
    \end{enumerate}
\end{enumerate}


\newpage
\section{Remote work}
One thing that can make life very convienent is learning how to work on remote
machines. CAEN has a large number of machines available for students to use
remotely, at \url{login.engin.umich.edu}. Let's check that out.

\texttt{ssh} is the \textbf{s}ecure \textbf{sh}ell program, and is used to
securely log in to remote machines.

Run \texttt{ssh login.engin.umich.edu} to log into a CAEN machine. Take a look
around, this is the same login environment and home directory as if you had
sat down at a physical CAEN machine.

Use \texttt{vim} or \texttt{emacs} to create a file called \texttt{test.txt}
on CAEN. Put some text in it, save, and quit.

Now try to run \texttt{gedit~test.txt}.

\textbf{What error do you get? Why?}
\begin{quote}
  \color{violet}
  \begin{lstlisting}
** (gedit:30002): WARNING **: Could not open X display

(gedit:30002): Gtk-WARNING **: cannot open display:
  \end{lstlisting}
  Graphical applications in Linux connect to a \emph{display server} called
  \textbf{X11}, commonly just called \textbf{X}. It's X that is responsible
  for hosting a \emph{window manager}, which is thing that lets a bunch of
  application windows all share the same screen.

  Because the machine you've logged into doesn't have a screen, there is no
  display server running, so \texttt{gedit} has nowhere to render itself,
  hence this error.

  One really cool thing about how X is designed: unlike OS X or Windows,
  programs that are running on one physical machine can display their window
  on a different physical machine, this is called \ul{X forwarding}.
\end{quote}

Close your ssh session (log out of the terminal, either by typing
\texttt{exit} or pressing \texttt{Ctrl-d} at an empty prompt). Then log in
again with the additional \texttt{-X} flag to ssh:
\texttt{ssh~-X~login.engin.umich.edu}.

Now try running \texttt{gedit test.txt} -- great! Only, now you cannot do
anything else in that terminal.

\textbf{How can you keep using this terminal without killing the gedit session
  that is already running?}

\textbf{How should you launch \texttt{gedit} so that you can still use the
  terminal?}

\emph{\small (Hint: remember we talked about shell \ul{job control} a few
  weeks ago)}

\begin{quote}
  \color{violet}
  You can keep using \texttt{gedit} by \ul{suspend}ing it by pressing
  \texttt{Ctrl-z} and then \ul{background}ing it by running \texttt{bg}.

  You can run \texttt{gedit} in the background when you first start it up by
  adding an ampersand to the end of the command, like this:

  \texttt{gedit \&}

  These \ul{job control} features are part of \texttt{bash}, though most
  shells include similar if not identical commands.
\end{quote}

\vfill
Try running a more graphically intensive program, such as the \texttt{eclipse}
IDE or \texttt{matlab}. How does it compare to sitting at a physical CAEN
machine? How does it compare to using
\href{http://caenfaq.engin.umich.edu/12374-Linux-Login-Service/how-do-i-connect-to-a-caen-linux-computer-remotely}{VNC}
to log into CAEN? (And how good is your Internet connection right now ;) ?)

\bigskip
\hrule
You may also find the \textbf{s}ecure \textbf{c}o\textbf{p}y program,
\texttt{scp}, useful for moving files between your machine and the remote
machine. Assuming you have the file `test.txt' on your local machine,

\texttt{scp~test.txt~login.engin.umich.edu:}

will copy the file to CAEN. Note the trailing colon, this is what tells
\texttt{scp} to copy to a remote machine.

You can also specify a path on one or both machines. Assuming that you have
the folder `inhomeoncaen' on your CAEN account and `local' in the current
directory of your machine,

\texttt{scp~login.engin.umich.edu:inhomeoncaen/test.txt~local/}

will the file \texttt{test.txt} from CAEN to your machine.

\texttt{scp} operates just like \texttt{cp}, that is, you have to pass the
\texttt{-r} flag if you want to copy a directory, such as

\texttt{scp~-r~project1~login.engin.umich.edu:classes/eecs280/}

\emph{(There's no question on \texttt{scp}, just a tool for you to play with)}

\end{document}
