\documentclass{article}
\usepackage[T1]{fontenc}

\usepackage{amssymb}
\usepackage{courier}
\usepackage{fancyhdr}
\usepackage{fancyvrb}
\usepackage[top=.75in, bottom=.75in, left=.75in,right=.75in]{geometry}
\usepackage{graphicx}
\usepackage{lastpage}
\usepackage{listings}
\lstset{basicstyle=\small\ttfamily}
\usepackage{mdframed}
\usepackage{parskip}
\usepackage{soul}
\usepackage{tabularx}
\usepackage{textcomp}
\usepackage{upquote}
\usepackage{xcolor}

% http://www.monperrus.net/martin/copy-pastable-ascii-characters-with-pdftex-pdflatex
\lstset{
upquote=true,
columns=fullflexible,
keepspaces=true,
literate={*}{{\char42}}1
         {-}{{\char45}}1
         {^}{{\char94}}1
}
\lstset{
  moredelim=**[is][\color{blue}\bf\small\ttfamily]{@}{@},
}

% http://tex.stackexchange.com/questions/40863/parskip-inserts-extra-space-after-floats-and-listings
\lstset{aboveskip=6pt plus 2pt minus 2pt, belowskip=-4pt plus 2pt minus 2pt}



\usepackage[colorlinks,urlcolor={blue}]{hyperref}
\usepackage[capitalise,nameinlink,noabbrev]{cleveref}
\crefname{section}{Question}{Questions}
\Crefname{section}{Question}{Questions}

\begin{document}

\fancyfoot[L]{\color{gray} C4CS -- F'16}
\fancyfoot[R]{\color{gray} Revision 1.0}
\fancyfoot[C]{\color{gray} \thepage~/~\pageref*{LastPage}}
\pagestyle{fancyplain}



\title{\textbf{Homework 6\\Debugging}}
\author{\textbf{\color{red}{Due: Saturday, October 22, 10:00PM (Hard Deadline)}}}
\date{}
\maketitle


\section*{Submission Instructions}
Submit this assignment on \href{https://gradescope.com/courses/2248}{Gradescope}.
You must submit every page of this PDF.
We recommend using the free online tool \href{https://www.pdfescape.com}{PDFescape}
to edit and fill out this PDF.
You may also print, handwrite, and scan this assignment.

There may multiple answers for each question. If you are unsure,
state your assumptions and your reasoning for why you think your answer
makes sense.


\newpage
\section{\texttt{Git}ting started}

When we first introduced git, we focused on the basics and how to use it
locally. Next lecture, when we revisit git, we'll talk about sharing a
repository with multiple collaborators and all that that entails. As a small
preview, we'll use a few of the features now. As we walk through this, we'll
introduce a few new terms and concepts. \emph{You are not expected to
understand all of this yet}, however we wanted to show you a real-world
example and give you some experience before the next git lecture.

First, we \texttt{\ul{clone}} a remote repository:

\begin{lstlisting}
@>@ git clone https://gitlab.eecs.umich.edu/c4cs/debugging-basics
\end{lstlisting}

Unlike \texttt{init}, which creates a new repository from scratch, the
\texttt{clone} command tells git to go copy an existing repository from the
supplied location. The location that you cloned from will automatically be set
up as the \texttt{\ul{\mbox{origin}}}. Much like in Makefiles, where there is
nothing actually special about the ``\texttt{all}'' target, there is nothing
magical about the \texttt{origin} in git. It is simply an arbitrary name used
by convention.

First things first, let's check out what status has to say about this new
repository.
\begin{lstlisting}
@>@ cd debugging-basics
@>@ git status
On branch master
Your branch is up-to-date with 'origin/master'.
nothing to commit, working directory clean
\end{lstlisting}
We have a new line of information!
Git is a \emph{distributed} version control system. This means that if you
make changes (add new commits) on your machine, they will not automatically
show up on any other machine.
Being ``up-to-date'' means that there are no changes on your machine that you
have not shared with the other machine, nor are there any changes on the other
machine that your machine knows about but has not integrated---this can be a
bit subtle and confusing, we'll talk about it next lecture.

Now let's take a quick look at what has happened in this repository:
\begin{lstlisting}
@>@ git log
\end{lstlisting}
Because we cloned a repository, it already has some history. You can also try
\texttt{git~log~-p} to see the changes in more detail.

Next we introduce a new feature, \texttt{\ul{branches}}:
\begin{lstlisting}
@>@ git checkout gdb-debug-1
Branch gdb-debug-1 set up to track remote branch gdb-debug-1 from origin.
Switched to a new branch 'gdb-debug-1'
\end{lstlisting}
{\small (don't worry about the tracking part for now)}

Branches allow us to take some code with a shared history and explore in two
different directions. Try running \texttt{git~log} again. Notice that the
first two commits are the same, but the third commit is different.

We can use the \texttt{\ul{git branch}} command, with no other arguments, to
simply list all of our local branches:
\begin{lstlisting}
@>@ git branch
  master
* gdb-debug-1
\end{lstlisting}
We can see from the asterisk that we currently have the \texttt{gdb-debug-1}
branch \texttt{\ul{checked~out}}.
By default, every repository begins with a branch named \texttt{\ul{master}}.
If you run \texttt{git~branch} in one of your other repositories, you will see
that it has one branch, named master, that you have been using all along.

You may find a visual view of what is going on here helpful:
\begin{lstlisting}
@>@ sudo apt-get install gitk
...
@>@ gitk --all
\end{lstlisting}
{\small (don't worry about the remote/origin labels for now)}

Notice how there is a shared history of two commits, that then diverges to two
branches, \texttt{master} and \texttt{gdb-debug-1}.

Each part of this homework will take place on a different branch.

\section{Finding primes}

The repository you have cloned is an attempt at a program to find prime
numbers. It will search from 3 up to a maximum number input by the user.

The program intends to follow the structure:
\begin{lstlisting}
  o prompt user for upper bound
  o for N=3..upper bound:
    - check if N is prime
      * if no prime n between 1..sqrt(N) divides N, then prime
    - save whether N is prime for future loops to use
    - if N is prime, print N
\end{lstlisting}

\subsection{Debugging with \texttt{gdb}}

\emph{You may find it helpful to consult the
  \href{http://c4cs.github.io/static/lecture/wk7-gdb.pdf}{gdb lecture notes}
  or this
  \href{https://ccrma.stanford.edu/~jos/stkintro/Useful_commands_gdb.html}
  {quick command reference}.
}



First, make sure you are on the \texttt{gdb-debug-1} branch:
\begin{lstlisting}
@>@ git branch
  master
* gdb-debug-1
\end{lstlisting}

Next, build and run the supplied program:
\begin{lstlisting}
@>@ make
@>@ ./prime
Find all prime numbers between 3 and ?
10
Segmentation fault (core dumped)
@>@ gdb -q ./prime
Reading symbols from ./prime...done.
(gdb) run
Starting program: /media/sf_prime/prime 
Find all prime numbers between 3 and ?
10

Program received signal SIGSEGV, Segmentation fault.
0x000000000040068f in check_prime (k=3) at check.c:15
@15      if (is_prime[j] == 1)@
\end{lstlisting}

\textbf{Explain in what case(s) executing this line of code could cause a
  segmentation fault?}
\vspace{3cm}

\textbf{What gdb commands could you run next to prove your hypothesis right or
  wrong?}
\vspace{3cm}


\newpage
% Fun fact: This was a "real" bug, that is, when I was writing this assignment
% I made a mistake and used the debugger to track it down
Now checkout \texttt{gdb-debug-2}:
\begin{lstlisting}
@>@ git checkout gdb-debug-2
@>@ make
@>@ ./prime
Find all prime numbers between 3 and ?
10
3 is a prime
5 is a prime
7 is a prime
@9 is a prime@
\end{lstlisting}
\textbf{Copy your debugging session and add notes explaining your
  thought process as you track down why this program thinks 9 is a prime.}\\
\emph{\small Hint: It looks like things go well up until 9. A good place to
  start then may be to break in and observe how the code determines whether 9
  is a prime number.}

\vfill
\hrule
{\footnotesize
This homework is based off of a simplified and updated version of this gdb
tutorial:
\url{http://heather.cs.ucdavis.edu/~matloff/UnixAndC/CLanguage/Debug.html}
If you would like some more practice, or simply a slightly different
explanation of the material, this may be a good source.
}


\newpage
\section{Valgrind}

Valgrind is a different kind of debugging tool. It is essentially a type of
virtual machine (like a simpler VirtualBox). Valgrind actually recompiles
every assembly instruction to allow it to intercept and monitor all of the
hardware calls made by the child process. While this is very powerful and
gives valgrind a lot of insight, it is also very slow, which can be
problematic for debugging large pieces of software.

Valgrind also can be challenging when libraries (such as the STL, or Boost) do
funny things with memory that result in a large number of false warnings. We
will look at valgrind in more depth during the second week on debugging
tools. For today, let's just explore the basic functionality and check out the
kind of things that valgrind can catch that gdb can't.

First, we'll need to install valgrind
\begin{lstlisting}
@>@ sudo apt-get install valgrind
\end{lstlisting}

Now, checkout the valgrind branch
\begin{lstlisting}
@>@ git checkout valgrind-debug
\end{lstlisting}

Notice (perhaps \texttt{gitk~-{}-all}) that the valgrind branch builds on the
master branch, with all the compiler warnings turned on. It has also
\ul{cherry-pick}ed the fix ``Stop searching for primes once sqrt(k) is
reached'', that is the same commit object that managed to end up on two
different histories.  This is one example of how git can be both very
powerful, and very confusing.

Finally, this branch adds a commit that consolidates they two source code
files into one and gets rid of all of the global variables in the process.
Unfortunately, this refactoring may also have inroduced a bug.

We run valgrind very similarly to gdb:
\begin{lstlisting}
@>@ make
@>@ valgrind ./prime
==11959== Memcheck, a memory error detector
==11959== Copyright (C) 2002-2015, and GNU GPL'd, by Julian Seward et al.
==11959== Using Valgrind-3.11.0 and LibVEX; rerun with -h for copyright info
==11959== Command: ./prime
==11959== 
Find all prime numbers between 3 and ?
10
3 is a prime@
==11959== Conditional jump or move depends on uninitialised value(s)
==11959==    at 0x400683: check_prime (prime.c:32)
==11959==    by 0x400739: main (prime.c:52)@
==11959== 
5 is a prime
7 is a prime
==11959== 
==11959== HEAP SUMMARY:
==11959==     in use at exit: 0 bytes in 0 blocks
==11959==   total heap usage: 0 allocs, 0 frees, 0 bytes allocated
==11959== 
==11959== All heap blocks were freed -- no leaks are possible
==11959== 
==11959== For counts of detected and suppressed errors, rerun with: -v
==11959== Use --track-origins=yes to see where uninitialised values come from
==11959== ERROR SUMMARY: 3 errors from 1 contexts (suppressed: 0 from 0)
\end{lstlisting}
{\small (The numbers in the left column are the process ID. They will be
  different for every person)}

\newpage
\textbf{Notice that ``3 is a prime'' printed before this warning. Why was this
  warning not emitted when running \texttt{check\_prime(3,~\dots)}?}
\vspace{2cm}

These tools really get powerful when you combine them. We can run valgrind in
a way that lets gdb connect to it, and allows us to debug/inspect whenever
valgrind detects a problem:
\begin{lstlisting}
@>@ valgrind --vgdb=yes --vgdb-error=0 ./prime
\end{lstlisting}
In another terminal, follow the directions that valgrind prints to connect
gdb. In this case, valgrind has already `run' the program for you and inserted
a breakpoint at the very beginning of the program, we just need to `continue'
it. The program will run until valgrind encounters an issue, at which point
valgrind will automatically break for you.
\begin{lstlisting}
@(gdb)@ continue

valgrind terminal:
==23589== Conditional jump or move depends on uninitialised value(s)
==23589==    at 0x400623: check_prime (prime.c:32)
==23589==    by 0x4006C4: main (prime.c:52)
==23589== 
@==23589== (action on error) vgdb me ...  # this is where valgrind waits for gdb@

gdb terminal:
Program received signal SIGTRAP, Trace/breakpoint trap.
0x0000000000400623 in check_prime (k=5, is_prime=0xffefff910) at prime.c:32
@32      if (is_prime[j] == 1)   # this is the problematic line of code@
\end{lstlisting}

~

\textbf{What is the value of \texttt{j} at this point?}\\
\vspace{1cm}

\textbf{What is the value of \texttt{is\_prime[j]} at the point?}\\
\vspace{1cm}

\textbf{Use git to look at the most recent commit. What line of code was
  deleted that should not have been?}



\end{document}
