\documentclass{article}
\usepackage{amssymb}
\usepackage{comment}
\usepackage{courier}
\usepackage{fancyhdr}
\usepackage{fancyvrb}
\usepackage[T1]{fontenc}
\usepackage[top=.75in, bottom=.75in, left=.75in,right=.75in]{geometry}
\usepackage{graphicx}
\usepackage{lastpage}
\usepackage{listings}
\lstset{basicstyle=\small\ttfamily}
\usepackage{mdframed}
\usepackage{parskip}
\usepackage{ragged2e}
\usepackage{soul}
\usepackage{upquote}
\usepackage{xcolor}

% http://www.monperrus.net/martin/copy-pastable-ascii-characters-with-pdftex-pdflatex
\lstset{
upquote=true,
columns=fullflexible,
literate={*}{{\char42}}1
         {-}{{\char45}}1
         {^}{{\char94}}1
}

% http://tex.stackexchange.com/questions/40863/parskip-inserts-extra-space-after-floats-and-listings
\lstset{aboveskip=6pt plus 2pt minus 2pt, belowskip=-4pt plus 2pt minus 2pt}

\usepackage[colorlinks,urlcolor={blue}]{hyperref}

\begin{document}


\fancyfoot[L]{\color{gray} C4CS -- F'17}
\fancyfoot[R]{\color{gray} Revision 1.1}
\fancyfoot[C]{\color{gray} \thepage~/~\pageref*{LastPage}}
\pagestyle{fancyplain}

\title{\textbf{Advanced Homework 10\\}}
\author{\textbf{\color{red}{Due: Wednesday, November 22nd, 11:59PM (Hard Deadline)}}}
\date{}
\maketitle


\section*{Submission Instructions}
To receive credit for this assignment you will need to stop by someone's
office hours, demo your running code, and answer some questions. \textbf{\color{red}{Make sure
to check the office hour schedule as the real due date is at the last office
hours before the date listed above.}} This applies to assignments that need to be gone over with a TA only.
\textbf{Extra credit is given for early turn-ins of advanced exercises. These details can be found on the website under the advanced homework grading policy.}

\medskip

\begin{mdframed}
\textbf{\color{red}{!!! IMPORTANT NOTE !!!}}

There is a known bug associated with the AFS portion of this exercise. If you
download a package during the AFS portion that crashes your Ubuntu machine, then
you are afflicted by the bug! :( Perform the following steps to see if the
bug would affect you and how to fix it ahead of time.

If the output of running \texttt{uname -r} is something that begins with
\texttt{4.8} you will likely need to follow one of the following actions:

\begin{enumerate}

    \item
        If you don't want to upgrade the Linux Kernel or encounter issues when
        upgrading the Kernel, you can also
        \href{https://www.ubuntu.com/download/desktop}{download and run Ubuntu
        17.04 as a new virtual machine.} This should come with a newer version
        of the Linux Kernel that will work for this assignment.

    \item
        If you are running Ubuntu \textbf{16.04}, run the following command to
        upgrade your version of the Linux Kernel.

\begin{lstlisting}
$ sudo apt-get install --install-recommends linux-generic-hwe-16.04 \
    xserver-xorg-hwe-16.04
\end{lstlisting}

        \textbf{This will probably take about 10-15 minutes to perform.}
        Afterwards, you will need to reboot and run \texttt{uname -r} to see if
        the upgrade took place properly.
\end{enumerate}


\end{mdframed}


\section*{The Andrew File System}

One thing you've hopefully noticed is that when you log on to any CAEN machine
(or ssh to any CAEN login server) all of your files are magically there.
That's because your files are stored using AFS, a network file system. When
you save files on a CAEN machine, they don't actually save to the local hard
disk, instead they go out over the Internet to a machine in a server room
somewhere.\footnote{Last I checked, for folks with uniqnames starting a-k,
this was the MITC Data Center on Oakbrook off of State. I don't recall where
the rest of the alphabet is stored. I think it was in the Southfield data
center at one point.} File reads similarly go over the network to grab the
data, so everything stays in sync automatically.\footnote{This glosses over
  some details about caching that's done for performance, but the basic idea
  is there.} Note that while this is similar to Dropbox, it is slightly
different as well since Dropbox stores and edits local copies on disk and then
syncs them in the background. That wouldn't work at the university scale,
imagine if every CAEN machine had to have enough disk space to hold every file
for every student! This means that AFS will only work if you are online.

Since AFS is a networked file system, you can also set up your local machine
to join the network.  This way, whenever you change anything on your machine,
it'll automatically update on CAEN and vice-versa. You could run your text
editor on your machine and your compiler on CAEN if you like.

\subsection*{Kerberos}

To get things working, there are two key things you need to get set up, the
first is \textbf{Kerberos}. Kerberos is a tool for authentication. You've been
using it every time you log into CAEN (or Cosign\footnote{
  You can actually set up your web browser to not need your password either.
  It can also use the same Kerberos ticket. I know this is do-able in Firefox,
  it may be do-able in other browsers as well.
}). With Kerberos, you must periodically (every 24~hours in umich's setup)
enter your password to get a new \emph{ticket}. Once you have this ticket on
your machine, AFS will use it to gain access to the files that you own.

\emph{Big Hint:} You will need a configuration file (\texttt{/etc/krb5.conf}).
Don't try to write this yourself! Instead login to a CAEN machine and grab a
copy from that machine.

\subsection*{AFS}

The next thing you'll need to install and set up is AFS.
Notice that once you install AFS you will have \emph{list} permissions on much
of the directory hierarchy. This will let you explore a little, such as seeing
that files exist, but not their contents.\footnote{You may notice the POSIX
permissions don't line up. That's because AFS uses
\href{https://web.njit.edu/info/afs.perms.html}{ACLs} to control permissions.
Run a \texttt{fs help} or \texttt{man fs} for more information. }
manage  For example, while holding credentials for \texttt{cgagnon}, I try to
look at thealex's (a past IA for the course) files, I can see the files in his
home directory, but not what's in them:

\begin{lstlisting}
  $ ls /afs/umich.edu/user/t/h/thealex
  ls: cannot access /afs/umich.edu/user/t/h/thealex/Shared: Permission denied
  ls: cannot access /afs/umich.edu/user/t/h/thealex/Private: Permission denied
  ls: cannot access /afs/umich.edu/user/t/h/thealex/Desktop: Permission denied
  ls: cannot access /afs/umich.edu/user/t/h/thealex/Downloads: Permission denied
  ls: cannot access /afs/umich.edu/user/t/h/thealex/Templates: Permission denied
  ls: cannot access /afs/umich.edu/user/t/h/thealex/Documents: Permission denied
  ls: cannot access /afs/umich.edu/user/t/h/thealex/Music: Permission denied
  ls: cannot access /afs/umich.edu/user/t/h/thealex/Pictures: Permission denied
  ls: cannot access /afs/umich.edu/user/t/h/thealex/Videos: Permission denied
  ls: cannot access /afs/umich.edu/user/t/h/thealex/hs_err_pid12482.log: Permission denied
  ls: cannot access /afs/umich.edu/user/t/h/thealex/Game.cpp: Permission denied
  ls: cannot access /afs/umich.edu/user/t/h/thealex/os_x_fuse_demo: Permission denied
  ls: cannot access /afs/umich.edu/user/t/h/thealex/rawr.cpp: Permission denied
  ls: cannot access /afs/umich.edu/user/t/h/thealex/repos: Permission denied
  ls: cannot access /afs/umich.edu/user/t/h/thealex/c4cs: Permission denied
  ls: cannot access /afs/umich.edu/user/t/h/thealex/krb5.conf: Permission denied
  ls: cannot access /afs/umich.edu/user/t/h/thealex/test.txt: Permission denied
  ls: cannot access /afs/umich.edu/user/t/h/thealex/484: Permission denied
  ls: cannot access /afs/umich.edu/user/t/h/thealex/dotfiles_old: Permission denied
  ls: cannot access /afs/umich.edu/user/t/h/thealex/oradiag_thealex: Permission denied
  $ cat /afs/umich.edu/user/t/h/thealex/test.txt
  cat: /afs/umich.edu/user/t/h/thealex/test.txt: Permission denied
\end{lstlisting}

If you find yourself getting a similar set of permission denied issues in your
own home directory, it means you are connected to AFS correctly, but you
haven't connected Kerberos and AFS correctly.


\subsection*{The Assignment}

\textbf{Get AFS installed and working on your local machine. You should have
read/write access to your own files.}

The writeup in this assignment gives you some intuition for what's going on
and the steps you'll have to take, but it is not a step-by-step guide -- there
are plenty of those out there (I even found a few umich-specific ones for the
Kerberos half, not that there's anything special you have to do to set things
up to work here, but they use umich as their example).

\emph{Note for users doing this on their Mac:
\href{https://www.openafs.org/}{OpenAFS} has not released a binary compatible
with the latest few macOS versions. I've seen some people use a commercial
package, but have not tried it myself.}

\subsection*{Submission checkoff}
\begin{itemize}
  \item[$\square$] Show the commands necessary to get AFS working locally
  \item[$\square$] Show that creating/modifying a file via the local AFS directory
      is also reflected when logged into CAEN
\end{itemize}


\subsection*{\texttt{Expect} Respect}

This sections touches on a handy scripting language called \texttt{expect} that
can be used to automate interactions with remote servers. What we'll be doing is
partially automating the login procedure for Michigan servers. We say partially
because of the two-factor step during the login procedure.

First you'll need to install \texttt{expect} for your system (Ubuntu and MacOS
both have packages available for download).

Some guiding steps this script should accomplish are:

\begin{enumerate}
    \item
        Prompt the user for their uniqname password and store
        it in a variable. This output should \textbf{not} be displayed on the screen when
        the user types in their password!
    \item
        Initiate a connection to CAEN and \texttt{send} the password previously
        provided by the user.
    \item
        Select one of the Duo options to fire off the push/call/text
        automatically.
\end{enumerate}

Although the creating of this script doesn't save too much time during the login
process for CAEN, this tool can be useful for a lot of other
tasks where you are using key-based access to servers and aren't prompted for
passwords or Duo options.

As always, there are multiple ways to accomplish some of the tasks above. A
potential solution requires no more than about 20 lines of \texttt{expect}
scripting.

\subsection*{Submission checkoff}
\begin{itemize}
  \item[$\square$] Explain why it's a bad idea to hardcode the password for the
      user into the expect script
  \item[$\square$] You'll almost certainly have \texttt{\textbackslash r} in your script. Explain
      what this means/does
  \item[$\square$] Show off your script working
    \begin{itemize}
        \item[$\square$] Prompting the user for their password (and not
            displaying it on the screen)
      \item[$\square$] Automatically selecting one of the Duo options
      \item[$\square$] Successfully logging the user into CAEN
    \end{itemize}
\end{itemize}

\end{document}
