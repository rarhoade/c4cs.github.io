\documentclass{article}
\usepackage[T1]{fontenc}

\usepackage{amssymb}
\usepackage{courier}
\usepackage{fancyhdr}
\usepackage{fancyvrb}
\usepackage[top=.75in, bottom=.75in, left=.75in,right=.75in]{geometry}
\usepackage{graphicx}
\usepackage{lastpage}
\usepackage{listings}
\lstset{basicstyle=\small\ttfamily}
\usepackage{mdframed}
\usepackage{parskip}
\usepackage{soul}
\usepackage{tabularx}
\usepackage{textcomp}
\usepackage{upquote}
\usepackage{xcolor}

% http://www.monperrus.net/martin/copy-pastable-ascii-characters-with-pdftex-pdflatex
\lstset{
upquote=true,
columns=fullflexible,
keepspaces=true,
literate={*}{{\char42}}1
         {-}{{\char45}}1
         {^}{{\char94}}1
}
\lstset{
  moredelim=**[is][\color{blue}\bf\small\ttfamily]{@}{@},
}

% http://tex.stackexchange.com/questions/40863/parskip-inserts-extra-space-after-floats-and-listings
\lstset{aboveskip=6pt plus 2pt minus 2pt, belowskip=-4pt plus 2pt minus 2pt}



\usepackage[colorlinks,urlcolor={blue}]{hyperref}
\usepackage[capitalise,nameinlink,noabbrev]{cleveref}
\crefname{section}{Question}{Questions}
\Crefname{section}{Question}{Questions}

\begin{document}

\fancyfoot[L]{\color{gray} C4CS -- F'16}
\fancyfoot[R]{\color{gray} Revision 1.0}
\fancyfoot[C]{\color{gray} \thepage~/~\pageref*{LastPage}}
\pagestyle{fancyplain}


\title{\textbf{Advanced Exercise -- Week 3\\}}
\author{\textbf{\color{red}{Due: Before October 8, 10:00PM}}}
\date{}
\maketitle

\section*{Submission Instructions}
To receive credit for this assignment you will need to stop by someone's
office hours, demo your running code, and answer some questions.

\section{Pretty \texttt{PS1}}

% I use a complex color here to see if folks won't find simpler expressions
% (hopefully)
Open a new terminal and try the following commands in order:

\begin{minipage}[t]{0.5\textwidth}
\begin{lstlisting}[numbers=left]
echo -e "\\033[44;3;70;38;5;214m"
<hit enter again>
PS1="Hello World -- "
echo -e "\\033[44;3;70;38;5;214m"
pwd
ls
pwd
echo -e "\\033[44;3;70;38;5;214m"
ls --color=none
pwd
reset
\end{lstlisting}
\end{minipage}
\begin{minipage}[t]{0.5\textwidth}
  \begin{lstlisting}[numbers=left,firstnumber=12]
pwd
bash --norc
echo -e "\\033[44;3;70;38;5;214m"
ls
pwd
ls --color=auto
pwd
exit
PS1="\\033[44;3;70;38;5;214mHello Again -- "
ls
\end{lstlisting}
\end{minipage}

\noindent
What happened to your terminal as you ran these commands?
Play around with some other forms of \texttt{PS1} and other colors, see what
happens. Why does calling \texttt{ls} sometimes reset things?

\medskip
\noindent
Create a custom \texttt{PS1} for yourself. Look into some of the options for
\texttt{PS1}, you will need to explain why you added the options you did and
decided against options you didn't choose.

\medskip
\noindent
Extend your PS1 by writing a bash function the changes your prompt in a way
that is not built-in to bash. Some examples: Add an asterisk if your are in a
git repository with uncommitted changes. Change the color if you are currently
in a shared directory (i.e. in a Dropbox folder). Change the color if the
current directory will not be saved across a reboot (i.e. if you are you are
somewhere in the \texttt{/tmp} directory).

\subsection*{Submission checkoff:}
\begin{itemize}
  \item[$\square$] Explain what \texttt{PS1} does
  \item[$\square$] Explain what you did to customize \texttt{PS1} and
    \textbf{why} you chose the customizations that you did.
    \begin{itemize}
      \item[$\square$] Explain how your custom function works.
    \end{itemize}
  \item[$\square$] Explain what the \texttt{PS2} variable controls. Change
    \texttt{PS2} from the default and show an example.
  \item[$\square$] Type \texttt{set -x}. Then type \texttt{ls}. Explain
    all of the output.
\end{itemize}


\section{Understanding tab completions}

\emph{This corner of the world is a little rougher and more complex. The goal
  of this task is to show you how you can use and even do some basic hacking
  on a tool that you don't completely understand. You \textbf{do not} need to
  develop a deep and complete understanding of bash completions for this task,
  you simply need to generate something that works.
}

\medskip
\noindent
Open a new terminal and try typing the following (note $<$tab$>$ means press
the tab key):

\begin{lstlisting}[numbers=left]
p<tab><tab>
y
<ctrl-c>
pi<tab><tab>
pin<tab><tab>
ping<tab><tab>
ping <tab><tab>
PATH=<enter>
p<tab><tab>
\end{lstlisting}

\medskip
\noindent
In addition to finding programs, tab completions can help you to use a program
correctly by hinting at what arguments a program accepts, try this:

\begin{lstlisting}[numbers=left,firstnumber=10]
# Open a new terminal (or manually set your PATH correctly again)
ping <tab><tab>
ping -<tab><tab>
ping -I<tab><tab>
ping -Q<tab><tab>
ping -Q 0 <tab><tab>
\end{lstlisting}

\noindent
Today, most programs include tab completion support, but this is a remarkably
manual process. Check out the contents of the
\texttt{/usr/share/bash-completion/completions/} directory.

\medskip
\noindent
Now take a look at \texttt{/usr/share/bash-completion/completions/ping}.
There's a lot going on in this example, but try to see if you can understand
some of how the completions are working. What do you think the result of
\texttt{ping~-T~$<$tab$><$tab$>$} will be?

\medskip
\noindent
(Hint: \texttt{||} in bash means ``if the previous thing failed, do the next
thing''. What's the output of \texttt{echo \$OSTYPE}? Will that equality pass
or fail?)


\subsection*{Writing our own completion file}

Completions are \emph{really} nice as a user. Find a project you've written
for a previous or current EECS class. What happens if you try to tab-complete
that project (e.g. \texttt{./project1 <tab><tab>})?

\medskip
\noindent
Let's try to make that more useful. Find a current or previous project that
takes at least two options (if you don't have one, write a trivial program,
such as \texttt{mycalc [--add,--subtract] NUM1 NUM2}). Be sure your project
executable is named something unique (e.g. there is already a built-in
\texttt{calc} program).

\medskip
\noindent
Write a custom completion file for your program that completes its arguments.
Install your completion such that it works whenever you open a new terminal.

\medskip
\noindent
\emph{%
  For this part, I recommend not starting from scratch. Find a simple
  completion file\footnote{
    Preview of coming attractions, try
    {\color{violet}\texttt{wc -l /usr/share/bash-completion/completions/* | sort -n | head}}
    to find a short, simple file to start with
  }
  to start from, hack it, mess with it, whatever until it does what you want.
}

\subsection*{Submission checkoff:}
\begin{itemize}
  \item[$\square$] Why is the list of tab completions different between lines
    1 and 9?
  \item[$\square$] What is the default tab completion behavoir for a program
    if no custom completion function has been written?
  \item[$\square$] Demonstrate your custom completion working.
    \begin{itemize}
      \item[$\square$] Explain what you had to do to ``install'' your
        completion file
      \item[$\square$] Explain what you had to do to get your completion file
        working
    \end{itemize}
\end{itemize}

\end{document}
