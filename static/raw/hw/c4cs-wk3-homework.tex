\documentclass{article}
\usepackage[T1]{fontenc}

\usepackage{amssymb}
\usepackage{courier}
\usepackage{fancyhdr}
\usepackage{fancyvrb}
\usepackage[top=.75in, bottom=.75in, left=.75in,right=.75in]{geometry}
\usepackage{graphicx}
\usepackage{lastpage}
\usepackage{listings}
\lstset{basicstyle=\small\ttfamily}
\usepackage{mdframed}
\usepackage{parskip}
\usepackage{soul}
\usepackage{tabularx}
\usepackage{textcomp}
\usepackage{upquote}
\usepackage{xcolor}

% http://www.monperrus.net/martin/copy-pastable-ascii-characters-with-pdftex-pdflatex
\lstset{
upquote=true,
columns=fullflexible,
keepspaces=true,
literate={*}{{\char42}}1
         {-}{{\char45}}1
         {^}{{\char94}}1
}
%\lstset{
%  moredelim=**[is][\color{blue}\bf\small\ttfamily]{@}{@},
%}

% http://tex.stackexchange.com/questions/40863/parskip-inserts-extra-space-after-floats-and-listings
\lstset{aboveskip=6pt plus 2pt minus 2pt, belowskip=-4pt plus 2pt minus 2pt}



\usepackage[colorlinks,urlcolor={blue}]{hyperref}
\usepackage[capitalise,nameinlink,noabbrev]{cleveref}
\crefname{section}{Question}{Questions}
\Crefname{section}{Question}{Questions}

\begin{document}


%%% TITLE INFO INSERTED HERE AUTOMATICALLY


\section{Understanding your \texttt{PATH}}

In a terminal, type \texttt{PATH=} (just hit enter after the equal sign, no
space characters anywhere). Try to use the terminal like normal (try running
\texttt{ls}). What happened?

\textbf{Give an example of a command that used to work but now doesn't:}
\vspace{1.5cm}

\textbf{Can you still run this command with an empty \texttt{PATH}? How?}
\vspace{1.5cm}

\textbf{Give an example of a command that works the same even with an empty
\texttt{PATH}. Why does this command still work?}
\vspace{1.5cm}


\section{Playing with the shell a bit: Special Variables}

Bash has quite a few special variables that can be very useful when writing
scripts or while working at the terminal.

\textbf{What does the variable \texttt{\$?} do? Give an example where this
value is useful}
\vspace{2cm}

\textbf{What does the variable \texttt{\$1} do? Give an example where this
value is useful}
\vspace{2cm}


\newpage
\section{Basic Scripting}

Recall from lecture that scripting is really just programming, only in a very
high-level language. Interestingly, \texttt{sh} is probably one of the oldest
languages in regular use today.

\texttt{make} is a good tool for build systems, but we can actually use some
basic scripting to accomplish a lot of the same things.
First, write a simple C program that prints ``Hello World!''.
Write a shell script named \texttt{build.sh} that performs the following
actions:
\begin{enumerate}
  \item Compile your program
  \item Runs your program
  \item Verifies that your program outputs exactly the string ``Hello World!''
    \begin{itemize}
      \item There are good utilities that check the \texttt{\ul{diff}}erence of two
        files. They could be helpful.
    \end{itemize}
  \item Prints the string ``All tests passed.'' If the output is correct, or
    prints ``Test failed. Expected output $>>$Hello World$<<$, got
    output $>>$\{the program output\}$<<$''.
\end{enumerate}

\textbf{Copy the output of \texttt{cat build.sh} here:}


\vfill

\section{Controlling your environment}

In lecture, we added a directory to our \texttt{PATH} so that we could just
type \texttt{hello} and the Hello World program would run. It would be
annoying to update the \texttt{PATH} variable every time we open a new
terminal. Fortuntately, we can do better.

\textbf{Describe how you would set up your system to modify your \texttt{PATH}
  automatically every time you open a new terminal (what file would you change
  and what would you put in it?)}
\vspace{3cm}

\textbf{Roughly how long did you spend on this assignment? \rule{5cm}{0.1mm}}
\end{document}
