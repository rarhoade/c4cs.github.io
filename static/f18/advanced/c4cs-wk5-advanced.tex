\documentclass{article}
\usepackage{amssymb}
\usepackage{courier}
\usepackage{fancyhdr}
\usepackage{fancyvrb}
\usepackage[T1]{fontenc}
\usepackage[top=.75in, bottom=.75in, left=.75in,right=.75in]{geometry}
\usepackage{graphicx}
\usepackage{lastpage}
\usepackage{listings}
\usepackage{lipsum}
\lstset{basicstyle=\small\ttfamily}
\usepackage{soul}
\usepackage{upquote}
\usepackage{xcolor}

\usepackage[colorlinks,urlcolor={blue}]{hyperref}

\begin{document}


\fancyfoot[L]{\color{gray} C4CS -- F'18}
\fancyfoot[R]{\color{gray} Revision 1.0}
\fancyfoot[C]{\color{gray} \thepage~/~\pageref*{LastPage}}
\pagestyle{fancyplain}

\title{\textbf{Advanced Homework 5\\}}
\author{\textbf{\color{red}{Due: Wednesday, October 17th, 11:59PM (Hard Deadline)}}}
\date{}
\maketitle


\section*{Submission Instructions}
To receive credit for this assignment you will need to stop by someone's
office hours, demo your running code, and answer some questions. \textbf{\color{red}{Make sure
to check the office hour schedule as the real due date is at the last office
hours before the date listed above.}} This applies to assignments that need to be gone over with a TA only.
\textbf{Extra credit is given for early turn-ins of advanced exercises. These details can be found on the website under the advanced homework grading policy.}


\section{Automated Background Testing}

As projects grow, the number and complexity of test cases grows as well. While
it's generally a good idea to run all of your test cases, it can be annoying
to sit around and wait for a long test case that tests a part of your program
that (you think) you didn't touch.

\medskip
\noindent
Scripting to the rescue! The goal is to write a script that runs tests in the
background. This script will make sure everything builds correctly, run all of
your tests, and, as a bonus, run an external correctness checker that can help
find mistakes.

\medskip
\noindent
Eventually, we'll invoke it like this: (calling the script with the repository to test as an argument)
\begin{lstlisting}
$ ./run-tests /Users/mmdarden/repos/eecs280-p1
\end{lstlisting}

\noindent
But first, read over everything your script should do:

\begin{enumerate}
    \item Create a directory in \texttt{/tmp} that is unique for this project if one does not already exist.
    \begin{itemize}
      \item \small \em Hint: The project's current directory is already unique
    \end{itemize}
  \item Assuming the directory you created above is at \texttt{\$PROJ\_TMP\_DIR}, run
    \begin{itemize}
      \item \texttt{git clone -{}-local \$1\ \$PROJ\_TMP\_DIR/\$(git describe -{}-always)}
    \end{itemize}
\item Call a \ul{second script} (\texttt{run-tests-bg} or something like that) that runs in the background to execute the
    remaining tasks. This second script should do all of its work in the clone
    you just made.
    \begin{itemize}
      \item \small \em Hint: You will likely need to pass arguments to this secondary script
    \end{itemize}
  \item Call \texttt{make} to build the project. You should save all of the
    regular output to a file named \texttt{build.log} and any errors to a file
    named \texttt{build.error}
    \begin{itemize}
      \item If the build fails, your script should stop and notify that the
        build failed. See the notify section below.
    \end{itemize}
  \item Run every test case for this project
    \begin{itemize}
      \item Your script should assume that any executable file with ``test''
        in the name is a test case. It should \textbf{not} hard-code a list of
        tests to run
      \item Your script should save the regular output and the error output of
        each test case to unique files.
      \item Your script should assume that test cases return 0 when they
        pass and non-zero otherwise.
    \end{itemize}
        \pagebreak
\item Generate a nice report when everything is done and alert the user
    using \texttt{notify-send}
        \begin{itemize}
            \item Try running \texttt{notify-send "I am a title" "And I am a body"}
            \item Your notification should include in the notification what you
                think is useful. At a minimum it should include how many test
                cases passed.
        \end{itemize}
\end{enumerate}

\medskip
\noindent
Notice that this script is \textbf{generic}. It should work in any repository.

\bigskip
\hrule
\bigskip

\medskip
\noindent
To get started, use the EECS\,280~W15 repository you created for Homework~2. If you did not do homework 2, follow the steps on page 2 of Homework 2 to get the starter files for this assignment and set up a git repository.

\medskip
\noindent
Create a directory to hold the scripts you're working on
\begin{itemize}\tt
  \item mkdir \textasciitilde/scripts
\end{itemize}
This is going to be a rather complicated script, so it's a good
idea to put it under version control as well.
\begin{itemize}\tt
  \item cd \textasciitilde/scripts
  \item git init
\end{itemize}
Inside this directory, create a file named \texttt{run-tests} with the
following contents.  Also be sure to make this script executable.
\begin{lstlisting}
#!/usr/bin/env bash

echo "Hello, I am your script running."

echo "Number of arguments I received: $#"
echo "Argument[0] (the program being run): $0"
echo "Argument[1]: $1" # This is empty if no arguments were passed
echo "Argument[2]: $2" # This is empty if only one argument was passed

# Notice what directory this script is executed from. This is important
# when you try to call your helper script
echo "$(pwd)"

sleep 10s && echo "It is annoying to wait for long commands to finish"

sleep 10s & echo "We can put them in the background so we don't have to wait"

notify-send "Test Message" "Blocking part of the script finished"

echo "Notice that the second sleep is still running, we can tell by: $(pidof sleep)"
\end{lstlisting}
%
Don't forget to add and commit the starter code
\begin{itemize}\tt
  \item git add run-tests
  \item git commit
\end{itemize}
%
Now, let's try it out with the repository you created in Homework~4:
\begin{itemize}\tt
  \item \textasciitilde/scripts/run-tests \textasciitilde/eecs280-w15~~~\#\,Or wherever you put this
\end{itemize}

\medskip
\noindent

\newpage

\subsection*{Some final tips}

\begin{itemize}
    \item \hl{Build up your solution in parts!} Get step 1 working then commit it.
    Then onto step 2.
  \item Steps 1 and 5 are probably the most challenging.
  \item You, of course, do not need to actually implement project~2 from
    EECS\,280.  I recommend something like this:%
\lstset{basicstyle=\footnotesize\ttfamily}
\begin{lstlisting}
$ git log --oneline -n1 -p
dfaeb04 Force filter_test to always succeed
diff --git a/filter_test.cpp b/filter_test.cpp
index eebeb87..fa1bc1e 100644
--- a/filter_test.cpp
+++ b/filter_test.cpp
@@ -20,6 +20,7 @@ bool isPrime(int x)

 int main()
 {
+    return 0;
     int numbers[] = { 3, 20, 46, 43, 9, 17, 103, 102 };
     const int numSize = sizeof(numbers) / sizeof(int);
     int primes[] = { 3, 43, 17, 103 };
\end{lstlisting}
\end{itemize}

\subsection*{Submission checkoff}
\begin{itemize}
  \item[$\square$] Explain what the clone command from step~2 does.
  \item[$\square$] Show off your script working
    \begin{itemize}
      \item[$\square$] With all passing test cases
      \item[$\square$] With some failing test cases
      \item[$\square$] Show how to find and examine the output from a failing
        test case
    \end{itemize}
  \item[$\square$] Show that your script is generic by running it with different
    git repositories that contain tests
    \begin{itemize}
      \item If you aren't in a programming EECS class, grab
        \href{https://eecs280staff.github.io/eecs280.org/}{any other 280
        project}.
    \end{itemize}
\end{itemize}


\medskip

\subsection*{[Optional Section] Automating Automated Background Testing}
Surprise! The way you wrote your script makes it very easy to install it as a git
\texttt{post-commit} hook. However, one more change will need to be made to the
script. The \texttt{post-commit} hook doesn't pass in the repository path as the
argument to the script. Instead, you will need to check in your script if the
repository path is given to the script. Based on that information you know
whether the script is run by git or by a user (since a user would pass in the
repository path). If it's run by git, you'll need to know what directory the
script is run from. Some info from the
\href{https://git-scm.com/docs/githooks}{git docs} contains this information:

\begin{quote}
    Before Git invokes a hook, it changes its working directory to either
    \texttt{\$GIT\_DIR} in a bare repository or \textbf{the root of the working tree in a
    non-bare repository.}
\end{quote}

\noindent
Emphasis mine.

\begin{itemize}
    \item Hint: \texttt{pwd} might come in handy here
\end{itemize}

\medskip

\noindent
Once those changes have been made, try:

\begin{itemize}\tt
    \item cd \textasciitilde/eecs280-w15/p2/.git/hooks~~~\#\,Or wherever you put this
    \item ln -s \textasciitilde/scripts/run-tests ./post-commit
\end{itemize}

\noindent
The command \texttt{ln -s} creates a symbolic link, or symlink. Recall
symlinks are just like pointers. This design lets us install the same commit
hook into multiple repositories just by pointing to it. If we ever update or
make improvements to the hook, all of the repositories that use it will
automatically get upgraded.

\medskip

\noindent
All right, we are finally set up. Make a new commit and test out the new hook!

\end{document}
