\documentclass{article}
\usepackage[T1]{fontenc}

\usepackage{amssymb}
\usepackage{courier}
\usepackage{fancyhdr}
\usepackage{fancyvrb}
\usepackage[top=.75in, bottom=.75in, left=.75in,right=.75in]{geometry}
\usepackage{graphicx}
\usepackage{lastpage}
\usepackage{listings}
\lstset{basicstyle=\small\ttfamily}
\usepackage{mdframed}
\usepackage{parskip}
\usepackage{soul}
\usepackage{tabularx}
\usepackage{textcomp}
\usepackage{upquote}
\usepackage{xcolor}

% http://www.monperrus.net/martin/copy-pastable-ascii-characters-with-pdftex-pdflatex
\lstset{
upquote=true,
columns=fullflexible,
keepspaces=true,
literate={*}{{\char42}}1
         {-}{{\char45}}1
         {^}{{\char94}}1
}
%\lstset{
%  moredelim=**[is][\color{blue}\bf\small\ttfamily]{@}{@},
%}

% http://tex.stackexchange.com/questions/40863/parskip-inserts-extra-space-after-floats-and-listings
\lstset{aboveskip=6pt plus 2pt minus 2pt, belowskip=-4pt plus 2pt minus 2pt}



\usepackage[colorlinks,urlcolor={blue}]{hyperref}
\usepackage[capitalise,nameinlink,noabbrev]{cleveref}
\crefname{section}{Question}{Questions}
\Crefname{section}{Question}{Questions}

\begin{document}


\fancyhead[C]{\hl{Select the right page in Gradescope or we will not grade the question!}}
\fancyhead[L]{}
\fancyhead[R]{}

\fancyfoot[L]{\color{gray} C4CS -- F'18}
\fancyfoot[R]{\color{gray} Revision 1.0}
\fancyfoot[C]{\color{gray} \thepage~/~\pageref*{LastPage}}
\pagestyle{fancyplain}

\title{\textbf{Homework 2\\Git}}
\author{\textbf{\color{red}{Due: Wednesday, September 26th, 11:59PM (Hard Deadline)}}}
\date{}
\maketitle


\section*{Submission Instructions}
Submit this assignment on \href{https://gradescope.com/courses/24368}{Gradescope}.
You may find the free online tool \href{https://www.pdfescape.com}{PDFescape}
helpful to edit and fill out this PDF.
You may also print, handwrite, and scan this assignment.


\section{Git's Chicken and Egg Problem}

Git has a (well-deserved) reputation as a tool with a high learning curve.
There are several reasons for this. One is that git has a bad user interface
-- the commands and flags you use to get things done are frustratingly
inconsistent and simply must be learned over time.
The bigger reason, however, is that using git requires having an understanding
of how git works and learning how git works requires having some experience
using git. This circular dependency makes it hard to get started as there is
no good way to break in. The best we can do is try to hold your hand as you
get started.

\medskip
\noindent
\textbf{Complete the first two main levels within the "Main" section at \url{https://learngitbranching.js.org/}.}


These levels are titled "Introduction Sequence" and "Ramping Up" respectively, and both levels have 4 mini challenges.
As you complete these mini challenges, the icon will be replaced with a green checkmark. To prove you have completed portion of the assignment,
take a selfie with your laptop and upload that image to Gradescope along with your submission.

\subsection*{Some other tutorials / resources}

There is a \emph{lot} written on the Internet about how to learn git. Some
highlights:

\begin{itemize}
  \item For a deeper dive into git from the basics to the very advanced, check
    out \href{https://www.atlassian.com/git/tutorials}{Atlassian's tutorials}
    (Atlassian is the company behind BitBucket and JIRA).
  \item For more visual learners, check out this
    \href{http://pcottle.github.io/learnGitBranching/}{interactive visual
    learning game} -- caveat this game simplifies some things, but I think
    it's still a useful tool to learn from.  There's even some
    \href{https://github.com/pcottle/learnGitBranching/issues/201}{low-hanging
    fruit} lying around to improve this tool.
  \item A lighter, more humorous
    \href{http://adriansampson.net/blog/git.html}{``quick-reference'' guide}
    and \href{http://ohshitgit.com/}{a more, uh, frustrated version} (warning:
    language).
\end{itemize}


\newpage
\section{Using git in your own projects}
\textbf{\color{red} This sets up the \ul{Winter 2015} EECS 280 P2 as
an example. 280 students please don't get confused!!}
\begin{enumerate}
    \setcounter{enumi}{-1}
  \item Install git: \texttt{sudo apt install git}
\end{enumerate}
\noindent
Using version control well is a habit learned over time. As a first step, for
the rest of the term, the \textbf{first} thing you should do whenever an EECS
class assigns you a project is:
\begin{enumerate}
  \item Create a repository
    \begin{Verbatim}[fontsize=\footnotesize]
    > mkdir -p eecs280-w15/project2
    > cd eecs280-w15/project2
    > git init
    \end{Verbatim}
  \item Grab a copy of the spec
    \begin{Verbatim}[fontsize=\footnotesize]
    > wget 'https://drive.google.com/uc?id=0B4qlH840ZwikRmQtdkRIeTZjODA&export=download' -O spec.pdf
    \end{Verbatim}
  \item Grab any starter code
    \begin{Verbatim}[fontsize=\footnotesize]
    > wget 'https://drive.google.com/uc?id=0B4qlH840ZwikbkZLS3Z5YTVSeW8&export=download' -O eecs280-w15-p2.tgz
    > tar -xf eecs280-w15-p2.tgz
    > rm eecs280-w15-p2.tgz
    \end{Verbatim}
  \item Create a blank starter cpp file
    \begin{lstlisting}
    > touch p2.cpp
    \end{lstlisting}
  \item Add it all to git and commit it
    \begin{lstlisting}
    > git add *
    > git status # Check to make sure everything looks right
    > git commit
    \end{lstlisting}
\end{enumerate}
Do this \textbf{before} even beginning to read the spec. Make it a habit.

\medskip
\noindent
If you try typing \texttt{make}, you'll find that this starter code does not
compile. The spec has a great tip at the bottom of page~11 that suggests
adding stub functions. {\small (No, you don't actually need to read this spec)}

\medskip
\noindent
There are 15 functions we need to stub out. Fortunately, your c4cs staff have
saved you some trouble. You should download our copy of \texttt{p2.cpp}:

\begin{Verbatim}[fontsize=\footnotesize]
> wget 'https://raw.githubusercontent.com/the-alex/c4cs-eecs280/master/p2.cpp' -O p2.cpp
\end{Verbatim}

\noindent
Now try running \texttt{make}. Cool; the project builds. But what changed?
Fortunately, git can help us figure this out.

\textbf{What command will show you the \hl{difference} between your working
directory and already committed changes?}

\vspace{1.0cm}
\noindent
\textbf{Now, go ahead and commit the changes to \texttt{p2.cpp} \underline{only}.}

\newpage

\medskip
\noindent
You ran \texttt{git status} before committing those changes, right? (Because
it's \emph{always} a good idea to run \texttt{git status} \emph{all} the time).
You probably noticed that annoying ``Untracked Files'' section that
lists all of the compiled output the Makefile made. It doesn't really make
sense to track compiled output in a version control system (it is hard to
track changes on a binary file such as an executable).\footnote{%
  Some binary files are good to check in though, such as the project
  specification, which will rarely or never change.
}
Instead, we would prefer if git would \hl{ignore} the built output.

\medskip
\noindent
\textbf{Commit a change so that \underline{built output is \hl{ignored}}.} When you are done,
\texttt{git status} should print
\begin{Verbatim}[fontsize=\footnotesize]
  On branch master
  nothing to commit, working directory clean
\end{Verbatim}

\bigskip
\textbf{Once you have completed all of the previous steps, copy
the output of:\\
~\\
\texttt{$>$ git log -n2 -p | head -n 40}}

\vfill

\subsection{Preparing for the future}
\noindent
This question summarizes the setup and simple usage loop for git. Make some
changes, add them, and commit them. If you need to delete a file, use
\texttt{git~rm~file\_to\_delete} instead of just \texttt{rm}. From this moment
onward, you should use git for all of your EECS projects. Seriously. Use it.

Set up git repositories for all of your other EECS projects this term. If you
are currently in the middle of a project, create an initial commit at its
exact state right now (even though it is not yet finished / not working).
\textbf{\large \color{red}
  \begin{center}\texttt{* * *}\end{center}
  Later in the term we will analyze how you have been using git and how you
  may improve. It is important that you have \hl{at least one non-trivial
  project} (e.g. a class project) that you have used git to maintain.
  \begin{center}\texttt{* * *}\end{center}
}
\noindent
Repositories from group projects are fine. If you use a different version
control system (\texttt{svn, bzr, hg, darcs, cvs, rcs, \dots}), choose one
project to manage with git instead, it's worth learning, you will encounter it.

\newpage
\section{A little about shells and your \emph{environment}}

Try the following:

\begin{lstlisting}
# The /tmp directory holds temporary files. It's a great place to do some
# quick tests or experiments, however it is automatically emptied every time
# the machine reboots. This means if your machine crashes, you *will* lose
# everything in /tmp.
#
# Don't ever put anything you care about in /tmp !!

> mkdir /tmp/throwaway
> cd /tmp/throwaway
> git init
> echo hello > world
> git add world
> git commit
\end{lstlisting}

\noindent
Git automatically opens a text editor for you to enter a commit message.

\bigskip
\noindent
\textbf{What text editor did git open for you? \rule{5cm}{0.1mm}}

\bigskip
\noindent
Git is not the only program that will open an editor automatically.
If you don't like the default choice, we can change it using an
\emph{environment variable} called \texttt{EDITOR}. Try the following:

\begin{lstlisting}
> echo "it's a small" > world
> git add world
> export EDITOR="emacs -nw"
> git commit
\end{lstlisting}

{\small\emph{Note: Shells are fussy about spaces, \textbf{do not} put spaces
    around the \texttt{=} in these commands}}

\bigskip
\noindent
\textbf{What text editor did git open for you? \rule{5cm}{0.1mm}}

\bigskip
\noindent
\textbf{What does the \texttt{-nw} flag do?
  {\footnotesize (Hint: Try it without \texttt{-nw})}
}
\vspace{1.0cm}


\noindent
Perhaps, however, you do not want to change the editor used for every program,
but only for git. We can do that too. Try the following:

\begin{lstlisting}
> echo "on top of the" > world
> GIT_EDITOR=gedit git commit world
\end{lstlisting}

\noindent
\textbf{If you set \texttt{EDITOR} and \texttt{GIT\_EDITOR} to different
values, which takes priority?}
\vspace{1.0cm}

\vfill

\noindent
\textbf{Roughly how long did you spend on this assignment? \rule{5cm}{0.1mm}}
\medskip

\hrule
\small
Some extra things to think about {\small (i.e.\ not graded)}:
Notice that we used two different approaches for setting environment
variables.
In the second example, we set the environment variable in-line, which means it
only lasts for the lifetime of that command. If you run \texttt{git\,commit}
again without the \texttt{GIT\_EDITOR=} part, git will use the old editor.

In the first case we used the \texttt{export} command, which sets
that variable permanently for the lifetime \emph{of that shell}. If you make
another commit in that shell, git will use the same value of \texttt{EDITOR}.
What happens if you open a new terminal? Which editor gets used?

It would be annoying to manually set \texttt{EDITOR} every time you open a new
shell. When \texttt{bash} (the shell program) starts, it reads the file
\texttt{\textasciitilde/.bashrc}, that is the ``hidden file'' (leading
\texttt{.}) named \texttt{.bashrc} in your home directory
(\texttt{\textasciitilde} is a shortcut to your home).
Try adding \texttt{export EDITOR="gedit"} to your
\texttt{\textasciitilde/.bashrc} file and then make some new commits. Does the
editor in your \emph{current} shell change? What about if you open a
\emph{new} shell? What if you don't include the ``\texttt{export}''?


\end{document}
